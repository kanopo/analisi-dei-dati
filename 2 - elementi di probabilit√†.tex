

\chapter{Elementi di probabilità}

Ci sono due interpretazioni, \textbf{frequentista} o \textbf{soggettivista}, nel'interpretazione 
frequentista si ha che la probabilità è una proprietà dell'esito stesso dell'esperimento
e si ricava ripetendo l'esperimento.(scienza)

Nel caso di interpretazione soggettivista la probabilità viene precisata in base alla fiduci dell'esito.(filosofia e finanza)

\section{Spazio degli esiti ed eventi}

\begin{itemize}
    \item \textbf{Spazio degli esiti($S$ oppure $\Omega$)} $=$ insieme degli esiti possibili
    \item \textbf{Eventi($E$)} $=$ insieme i cui elementi sono esiti possibili 
    \item \textbf{Unione eventi} $A\cup B=$ OR logico
    \item \textbf{Intersezione eventi} $A\cap B=$ AND logico, si indica molto spesso mediante una virgola
    \item \textbf{$A\cap B = \emptyset$} significa che sono due insiemi disgiunti o mutualemnte esclusivi
    \item \textbf{$A^B$} significa complementari
    \item \textbf{$A \subset B$} significa che $A$ è un sottoinsieme di $B$
\end{itemize}


\section{Assiomi della probabilità}

\begin{enumerate}
    \item La probabilità di un evento $E$ è: $0 \leq P(E) \leq 1$
    \item La probabilità dello spazio degli esiti $P(S)$ è $1$  
    \item La probabilità che si verifichi almeno un eventi di un insieme di eventi mutualemnte esclusivi è uguale alla somma delle loro probabilità
    $P(\displaystyle\cup_{i = 1}^n E_i) = \displaystyle\sum_{i = 1}^{n} P(E_i)$
\end{enumerate}

\section{Spazio esiti equiprobabili}

Si capisce molto facilmete guardando la probabilità di lanciare un dato e le facce hanno tutte la stesa probabilità
di uscire. 

$S = \{1, 2, \dots, n\}$

La probabilità $P = \displaystyle\frac{1}{n}$

\subsection{Coefficiente binomiale}

Voglio determinare il numero di diversi gruppi di $r$ oggetti che si possono formare scegliendoli da 
un'insieme di $n$. 

\begin{equation}
    \binom{n}{k} = \displaystyle\frac{n!}{r!(n-r)!}
\end{equation}


\section{Probabilità condizionata}

La probabilità condizionata di $E$ dato $F$ è:
\begin{equation}
    P(E|F) = \displaystyle\frac{P(E\cap F)}{P(F)} 
\end{equation}


\section{Fattorizzazione di un evento e formula di Bayes}


\begin{equation}
    E = (E \cap F) \cup (E \cap F^C)
\end{equation}

\begin{equation}
    \begin{split}    
    P(E) = P(E \cap F) + (E \cap F^C) \\ 
        = P(E|F)P(F) + P(E|F^C)P(F^C) \\
        = P(E|F)P(F) + P(E|F^C)[1 - P(F)]
    \end{split}
\end{equation}

\begin{itemize}
    \item $E$ intersezione $F$ sarebbe un and logico 
    \item $E$ intersezione $F^C$ risulta nell'insieme $E$ togliendo le parti in comune con $F$
\end{itemize}




\section{Eventi indipendenti}

Se due eventi $E$ ed $F$ sono indipendenti, allora
\begin{equation}
    P(E\cap F) = P(E)P(F) 
\end{equation}

Sono indipendenti e significa che la riuscita o meno di un evento non condiziona la riuscita dell'altro. 

La formula generale \`{e}:
\begin{equation}
    P(\cap_{i = 1}^r E_{ai} = \prod_{i = 1}^{r} P(E_{ai})) 
\end{equation}
