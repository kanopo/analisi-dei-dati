\chapter{La legge debole dei grandi numeri}
La legge debole dei grandi numeri afferma che, se si considera una sequenza di variabili aleatorie indipendenti ed identicamente distribuite $X_1, X_2, \ldots, X_n$, la media campionaria $\bar{X_n}$ converge in probabilità al valore atteso della variabile aleatoria, ovvero:

$$\lim_{n \to \infty} P(|\bar{X_n} - \mu| < \epsilon) = 1$$

dove $\mu$ è il valore atteso di $X_i$ e $\epsilon$ è un valore positivo arbitrario. In altre parole, la probabilità che la media campionaria si discosti dal valore atteso di più di un valore prefissato $\epsilon$ diventa sempre più piccola all'aumentare del numero di osservazioni $n$.

Questa legge fornisce una garanzia formale dell'andamento della media campionaria al crescere del numero di osservazioni e rappresenta un importante risultato in teoria della probabilità e nelle applicazioni pratiche dell'analisi dei dati.
